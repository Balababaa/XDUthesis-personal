%!TEX root = ../Demo.tex
\begin{thanksfor}
    % 四年一晃而过,期间有期待、兴奋、惊喜交集,也有焦虑、恐惧、忐忑不安,
    % 但唯一不变的是我对未来生活的憧憬和思考。完成毕业设计,离开生活了四年的学校,是一个终点,
    % 但也是另一个起点!

    % 首先,我要感谢西电的所有老师,能成为他们的学生是我的荣幸。高等数学、线性代数和离散数学
    % 教会我如何的理性思考问题;马原、毛概和形势与政策,教会我如何用哲学的思维理解世界。
    % 毕业设计的指导老师和学长在我遇到困惑的时候,都会积极解答。

    % 其次,我要感谢在公司实习中认识的导师、同事和小伙伴。刚踏入职场的我,对工作可以说是一窍不通,感谢导师
    % 和同事给了我足够的时间去适应和熟悉工作环境。在他们的帮助和指导下,我慢慢的了解如何工作,跟上了其他人的节奏。

    % 然后,我要感谢母校西电的栽培,愿西电越办越好,不断提高国际影响力,在电子信息、计算机科学与技术等方向上
    % 越做越好。依稀记得我刚刚进入校园的时候,不论做什么事,我都畏手畏脚,不敢多问,不敢多说。而四年之后的我,
    % 变得更加的开朗,这离不来西电四年陪伴。

    % 再次,我要感谢西安这座历史底蕴丰厚的城市,它见证了我从一个稚嫩的人,成长为能肩负责任的男子汉。
    % 多少次往返于杭州和西安,行走在承载历史的古城墙,徜徉于大明宫、大雁塔和秦始皇陵。
    % 我求学的足迹,永远的留在了这片土地上。

    % 最后,我要感谢家人这四年来一如既往对我的关心和照顾,最坚实的精神支柱其实家人的关心。
    % 虽然不在他们身边,但是还是能感受到他们对我深切的关心。
    % 你们的健康和快乐是我今生最大的愿望。

\end{thanksfor}