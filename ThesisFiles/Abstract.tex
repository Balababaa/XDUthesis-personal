%!TEX root = ../Demo.tex
% 中文摘要
\begin{abstract}
  随着近年来各类智能设备数量的爆发性增长,视频监控系统正朝着数字化、智能化、网络化、人性化的方向发展。
  基于此,本文设计并实现了一个基于Spring Boot的视频实时监控系统。
  本文首先介绍了视频监控系统的相关概念、研究现状和发展趋势。
  然后介绍了流媒体传输技术的有关概念,分析比较了RTMP、HTTP-FLV和HLS这三个常用的流媒体传输协议,
  对三者的格式和原理进行了的阐述,并讨论了它们在实际使用过程中各自的利弊,最终决定使用 HTTP-FLV 进行系统开发。
  此外也介绍了基于Java的Spring Boot开发框架的使用方法,并对其特点、设计理念和功能做了细致的阐述。
  % 此外,也介绍了对象拷贝技术MapStruct、基于内存的数据库Redis和基于RBAC的权限系统。
  % 最后对实时监控视频系统的发展趋势作了分析。
  最后,本文详细的阐述了本系统的总体设计和实现细节。
  在此次毕业设计的任务中,运用到了Spring Boot、Redis、MapStruct和Vue.js等技术,逐步完成了视频实时监控系统的技术方案设计。
  最终实现了一个基于Spring Boot的视频实时监控系统,具备实时监控视频查看、监控视频保存和检索等功能。

\end{abstract}
\keywords{流媒体传输技术, Spring Boot, HTTP-FLV, 实时视频监控系统}

% 英文摘要
\begin{enabstract}
  With the explosive growth in the number of various types of smart devices in recent years, 
  video surveillance systems are developing in the direction of digitization, intelligence,
   networking, and humanization.
  Based on this, this paper designs and implements a real-time video surveillance system based on Spring Boot.
  This article first introduces the related concepts, research status and development trend of 
  the video surveillance system.
  Then introduced the related concepts of streaming media transmission technology, 
  analyzed and compared the three commonly used streaming media transmission protocols RTMP, HTTP-FLV and HLS,
   The format and principle of the three were elaborated, and their respective pros and cons in actual use were discussed.
    Finally, it was decided to use HTTP-FLV for system development.
    In addition, the use of the Java-based Spring Boot development framework is introduced, and its characteristics,
     design concepts and functions are elaborated.
    Finally, this paper elaborates on the overall design and implementation details of the system.
  In the task of this graduation project, technologies such as Spring Boot, Redis,
   MapStruct and Vue.js were used to gradually complete the technical scheme design 
   of the real-time video surveillance system.
  Finally, a real-time video monitoring system based on Spring Boot is realized, 
  with functions such as real-time monitoring video viewing, monitoring video storage and retrieval.
  % This paper first introduces the definition of streaming media transmission protocol, 
  % analyzes and compares the three commonly used streaming media transmission protocols RTMP, HTTP-FLV and HLS, 
  % conducts an in-depth analysis of the principles of the three, 
  % and discusses their pros and cons in actual use.
  % After that, the article introduces the use of Java-based Spring Boot development framework, and elaborates its characteristics, design concepts and functions in detail.
  % In addition, the object copy technology MapStruct, the memory-based database Redis and the RBAC-based permission system are also introduced.
  % Finally, the development trend of real-time surveillance video system is analyzed.

\end{enabstract}
\enkeywords{streaming media transmission protocol, Spring Boot, HTTP-FLV, real-time video surveillance system}