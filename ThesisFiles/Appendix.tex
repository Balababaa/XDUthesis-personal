%!TEX root = ../Demo.tex
\chapter{数据}
这里是附录的数据部分,其实是瞎写;来看个公式编号对不对,如式\eqref{Eq:ApEq1}所示。

\begin{equation}\label{Eq:ApEq1}
h_i = \frac{\max_{j=1}^{N}\{\frac{c_j}{f_j}\}-\frac{c_i}{f_i}}{\max_{j=1}^{N}\{\frac{c_j}{f_j}\}-\min_{j=1}^{N}\{\frac{c_j}{f_j}\}}
\end{equation}

\section{放松一下}
\subsection{惊鸿一面}

\begin{center}
\textbf{惊鸿一面}\cite{shanshui}

{\kaishu 许嵩}

\vspace*{1em}
翻手为云\quad 覆手为雨\\
金盆洗手止风雨\\
不恋红尘\quad 却难舍回忆\\
每一段都有你\\

\vspace*{.7em}
年少初遇\quad 常在我心\\
多年不减你深情\\
江山如画\quad 又怎能比拟\\
你送我的风景\\

\vspace*{.7em}
柳下闻瑶琴\quad 起舞和一曲\\
仿佛映当年\quad 翩若惊鸿影\\
谁三言两语\quad 撩拨了情意\\
谁一颦一笑\quad 摇曳了风景\\

\vspace*{.7em}
纸扇藏伏笔\quad 玄机诗文里\\
紫烟燃心语\quad 留香候人寻\\
史书列豪杰\quad 功过有几许\\
我今生何求\quad 唯你\\

\vspace*{.7em}
远山传来清晨悠然的曲笛\\
晓风掠走光影\\
残月沉霜鬓里\\
有了你\\
恩怨都似飞鸿踏雪泥
\end{center}

\subsection{无题}
\begin{center}
\textbf{无题}

{\kaishu 李商隐}

\vspace*{1em}
相见时难别亦难,东风无力百花残。\\
春蚕到死丝方尽,蜡炬成灰泪始干。\\
晓镜但愁云鬓改,夜吟应觉月光寒。\\
蓬山此去无多路,青鸟殷勤为探看。
\end{center}

\section{代码}
\lstinputlisting[language=C++,caption={C++ code},label=cpp]{./Code/CppTest.cpp}
%标题不编号
\lstinputlisting[language=Java,title={Java code},label=java]{./Code/JavaTest.java}
%无行号
\lstinputlisting[language=Matlab,style=nonumbers]{./Code/floyd.m}


\begin{thebibliography}{A1}
\bibitem[A1]{shanshui}
许嵩. 惊鸿一面[A]. 海蝶音乐. 不如吃茶去[C]. 北京: 北京海蝶音乐有限公司, 2014, 8.

\end{thebibliography}